
\section{Introducción}
\subsection{Motivación}
\par
Internet-of-Things (IoT) y las tecnologías Big Data han producido avances significativos en el dominio de los sistemas de gestión de flotas de vehículos. El paradigma IoT ha permitido mejorar el proceso de seguimiento y monitorización de vehículos y las técnicas Big Data son muy apropiadas para el análisis en tiempo real de la gran cantidad de datos obtenidos en este proceso\cite{1-1-3}. De este modo, han surgido nuevas aplicaciones de gestión de flotas que gestionan mejor los recursos de la empresa y ofrecen un procesamiento más sofisticado y escalable en sus diferentes escenarios.
\par
Las aplicaciones de gestión de flotas son, por tanto, un dominio apropiado para la aplicación de arquitecturas Big Data. El seguimiento de los vehículos genera un gran volumen de datos que, a través de diferentes técnicas de análisis de datos, podemos extraer información de gran interés para las empresas dedicadas a este sector. Dado que las aplicaciones de gestión de flotas, deben procesar esta gran cantidad de datos y proporcionar diferentes beneficios, las tecnologías Big Data son muy apropiadas para tratarlos. Algunos de los beneficios que podemos obtener con este tipo de aplicaciones es reducir el coste de gestionar la flota, ser más responsable con el medio ambiente y poder controlar cualquier tipo de robo o mal uso de los vehículos de la empresa propietaria de la flota \cite{1-1-1} \cite{1-1-2}.
\par
Movildata\footnote{\url{https://movildata.com/sobre-nosotros/}} es una empresa con sede en Murcia dedicada a ofrecer soluciones para la gestión de flotas. Esta empresa se ha integrado recientemente en Verizon Connect. Antes de llevarse a cabo esta integración, los tutores y alumno de este proyecto acordaron con Movildata desarrollar un proyecto piloto destinado a diseñar e implementar una arquitectura Big Data que se aplicase para ofrecer alternativas y nuevas funcionalidades a su solución de gestión de flotas. La empresa disponía de aplicaciones basadas en arquitecturas tradicionales con lo que el proyecto serviría como prueba de concepto de aplicación de tecnologías Big Data.
\par
\subsection{Objetivos}
\subsubsection{Objetivo principal}
El objetivo principal de esta tesis de máster ha sido la elección de una arquitectura big data para aplicaciones de gestión de flotas y su evaluación en un caso de estudio definido a partir de la información proporcionada por Movildata. Se realizará una prueba de concepto de la arquitectura que ayude a la empresa a conocer las nuevas tecnologías Big Data y cómo se podría beneficiar de su aplicación.
\par
\subsubsection{Objetivos secundarios}
Entre los objetivos secundarios encontramos los requerimientos típicos de las tecnologías Big Data. Por un lado, dicha arquitectura debe ser fácil de administrar y ampliar es decir, debe ser fácilmente escalable. En nuestro caso, buscamos reconocer la dificultad y capacidad de mantener estas tecnologías y la capacidad a tolerar fallos.
\par
A pesar de ser una prueba de concepto, tendremos que enfrentarnos a las dificultades de implementar y mantener la arquitectura. Además, veremos la capacidad de realizar nuevos desarrollos sobre la misma, comprobar la facilidad de reemplazar cualquiera de las herramientas que la componen y comprobar cómo funcionan. Por último tendremos valorar la capacidad de reemplazar a las tecnologías propuestas con las que se usaban tradicionalmente.
\par
Dado que el objetivo principal es Investigar las diferentes arquitecturas y tecnologías aplicables para el problema abordado, se deberán justificar las razones por las que hemos seleccionado determinadas herramientas. Por tanto, será necesario evaluar las distintas herramientas que nos ofrece el mercado.
\par	 	 	 	
Tras esto, decir que el hardware de desarrollo es limitado por lo que se debe encontrar la forma de exportar fácilmente las diferentes configuraciones. Por otro lado, nos ayudará a valorar si cumple los requisitos mínimos de rendimiento de las diferentes herramientas.
\par
Por último, se debe comprobar que la solución es escalable horizontalmente es decir, se escalará añadiendo más máquinas y no añadiendo más hardware al servidor. Dado esto se deberá usar una tecnología de virtualización suficientemente potente y ligera para poder añadir y quitar máquinas que proporcionen capacidad a la estructura. Por otro lado, la escalabilidad debe darse tanto en el almacenamiento como en procesamiento.
\par
\subsection{Metodología}

\subsection{Organización del documento}



