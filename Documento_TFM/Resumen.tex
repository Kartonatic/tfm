
\textsc{\noindent\LARGE Resumen}

Los sistemas de gestión de flotas de vehículos han evolucionado muy
rápidamente teniendo que gestionar gran cantidad de datos a día de
hoy. Gracias al paradigma Internet-of-Things (IoT) y el avance de las
tecnologías Big Data, podemos llegar a hacer frente al gran volumen de
datos que se deben procesar, dando respuestas en tiempo real en los
casos en los que sea necesario. Esto hace que realizar una plataforma
con herramientas pertenecientes a estos paradigmas sea muy apropiada
para los sistemas de gestión de flotas.

El objetivo principal de este Trabajo Fin de Máster ha sido la
elección e implementación de una arquitectura Big Data adaptada para
la gestión de flotas. Dicho esto, se realizará una prueba de concepto
de una arquitectura que ayude a las empresas del sector a conocer cómo
podrían beneficiarse de dicha implementación. Para esto, se ha
desarrollado una arquitectura Lambda capaz de realizar procesamiento
en tiempo real. En esta arquitectura se ha usado Apache Kafka para
mantener los mensajes de los vehículos en streaming; Apache Spark
para realizar procesamientos en tiempo real; Apache Hadoop como
almacenamiento a gran escala; MongoDB como base de datos geográfica,
y el Stack de Elastic para almacenar los datos procesados y mostrarlos
en un dashboard. Finalmente, se realizará una valoración de la
arquitectura propuesta.

\hfill \break \textbf{Palabras clave:} Big data; Gestión de flotas;
Arquitectura Lambda; Tiempo real; Escalado horizontal

\begin{otherlanguage}{english}

\textsc{\noindent\LARGE Abstract}

Vehicle fleet management systems have evolved rapidly, having to
manage a large amount of data currently. Due to the paradigm of the
Internet of Things (IoT) and the advancement of Big Data technologies,
we will be able to deal with the large volume of data that must be
processed, providing real-time responses in cases where it is
necessary. This allows a platform with tools belonging to these
paradigms to be very appropriate for fleet management systems.

The main objective of this Master's Thesis has been the election and
implementation of a Big Data architecture adapted for fleet
management. Therefore, a proof of concept of an architecture will be
carried out to help companies in the sector to know how they could
benefit from such an implementation. As a consequence of this, a
Lambda architecture, able to perform real-time processing, has been
developed. In this architecture, Apache Kafka has been used to keep
messages from vehicles in streaming; Apache Spark to perform
real-time processing; Apache Hadoop as large-scale storage; MongoDB
as a geographic database, and Stack of Elastic to store the processed
data and display it on a dashboard. Finally, an evaluation of the
proposed architecture will be carried out.

\hfill \break
\textbf{Key  words:} Big data; Fleet Management Systems; Lambda Architecture; Real-time; Horizontal scaling

\end{otherlanguage}

%%% Local variables:
%%% TeX-master: "main.tex"
%%% coding: utf-8
%%% ispell-local-dictionary: "spanish"
%%% TeX-parse-self: t
%%% TeX-auto-save: t
%%% fill-column: 75
%%% End:
