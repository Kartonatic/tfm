{\huge\noindent RESUMEN}

Los sistemas de gestión de flotas de vehículos, han evolucionado muy rápidamente teniendo que gestionar gran cantidad de datos a dia de hoy. Gracias al paradigma Internet-of-Things (IoT) y el avance de las tecnologías Big Data, podemos llegar  a hacer frente a este gran volumen de datos que se debe procesar, dando respuestas en tiempo real en los casos en los que sea necesario. Esto hace que una aplicación realizada con herramientas del conjunto de estos paradigmas, sea muy apropiada para los sistemas de gestión de flotas.

El objetivo principal de esta trabajo fin de máster ha sido la elección e implementación de una arquitectura big data para aplicaciones de gestión de flotas. Dicho esto, se realizará una prueba de concepto de la arquitectura que ayude a las empresas del sector a conocer las cómo podrían beneficiarse de dicha implementación.







\hfill \break
\hfill \break
\textbf{Palabras clave:} Big data; Gesión de flotas; Arquitectura Lambda; Tiempo real
\newpage
{\huge\noindent ABSTRACT}



\hfill \break
\hfill \break
\textbf{Key  words:} Big data; Fleet Management Systems; Lambda Arquitecture; Real time
