
\chapter{Validación\label{validacion}}

En este apartado vamos a analizar cómo se ha comportado la arquitectura
propuesta sobre el hardware que hemos usado. Evidentemente, los resultados
de tiempo de ejecución no corresponden con los de un cluster real pero, por
motivos de disponibilidad, no se ha podido realizar el estudio sobre uno de
estos.

La arquitectura propuesta nos ha aportado una flexibilidad que, con una
arquitectura tradicional, no tendríamos. En el caso de una arquitectura
Lambda tenemos dos vías para las cuales realizar procesamientos. Por un
lado tenemos la batch layer para los procedimientos más costosos y que
requieren obtener datos de histórico y por otro, la Speed layer que nos
permitirá obtener los procesamientos de tiempo real que nos sean
necesarios. También, dado que la funcionalidad no está focalizada en una
sola aplicación, sino que podemos distinguir varias partes, es posible
reemplazar cualquier herramienta, haciendo que sea así una tarea más
sencilla. Dado esto y, principalmente por poder separar los procesamientos
en tiempo real y de histórico, la arquitectura Lambda es más que
recomendable en sistemas de IoT que requieren gran cantidad de
procesamientos y, más aún, en las aplicaciones dirigidas al transporte y
gestión de flotas. Para terminar, decir que es capaz de soportar más cantidad
de mensajes que los usados en las pruebas, de hecho, se han hecho pruebas
sobre el hardware que disponemos con varios gigas de datos y es capaz de
llevarlo bien.

En cuanto a las herramientas utilizadas, comenzaremos analizando el uso de
Docker sobre esta arquitectura. Docker, nos ha aportado la capacidad de
mover las diferentes herramientas de un sitio a otro sin ningún problema
gracias a la encapsulación, lo que hará que sea más fácil poder probarlo en
un cluster en un futuro. Por otro lado, aunque la curva de aprendizaje es
dura al principio, luego se hace realmente fácil de manejar, aportando las
ventajas reflejadas en el apartado \ref{Docker}.

En cuanto a Apache Kafka, hemos encontrado un rendimiento más que notable,
siendo capaces de leer con varios procesos sobre un mismo Topic cuando
hemos realizado las diferentes pruebas. Además de esto, el hecho de que sea
tan fácil de configurar lo hace una herramienta muy recomendable. Una de
las propiedades que más llama la atención de Kafka y que lo hace tan
potente, es que podemos almacenar los mensajes durante un tiempo, lo que
permite que podamos leer gran cantidad de tramas en momentos determinados.
Esto hace que en aplicaciones de gestión de flotas sea muy recomendable, ya
que podemos buscar tramas erróneas en diferentes procesos de análisis que
se pueden ir ejecutando por otra parte.

Por la parte de Hadoop, encontramos que el tratamiento de datos es
complicado sin herramientas como Hive. Sin embargo, también hemos
encontrado la facilidad de integrarlo con diferentes herramientas. Por otro
lado, también vemos que el sistema de ficheros HDFS nos da la ventaja de
dejar de tener que usar RAIDs en nuestros servidores y poder acceder a los
datos de una forma rápida.

%RGM: AQUI SE DICE LA CANTIDAD DE MENSAJES QUE SOPORTA POR SEGUNDO, POR LA PARTE DE KAFKA, LA VERDAD ES LO QUE DIGO, ME SOBRA KAFKA PARA METER MENSAJES, DESDE MI PUNTO DE VISTA ES UNA DE LAS MEJORES HERRAMIENTAS, DE HECHO, PROBÉ A METER VARIOS GIGAS DE MENSAJES Y SE LOS CHUPABA BIEN CON MI PORTATIL
Por la parte de Spark, encontramos que las operaciones para obtener los
usuarios y comprobar si está cerca de un punto negro, se realizan muy
eficientemente y se pueden tratar las tramas rápidamente. Con las pruebas
realizadas hemos podido tratar más de 3000 tramas por debajo de los 10
segundos del microbaching. Otra observación a tener en cuenta es que la
primera vez tarda más debido a que tiene que cargar los datos de los
usuarios y las primeras iteraciones van desacompasadas. También encontramos
que, al mantener el índice de Kafka, si lo paramos y lo volvemos a
ejecutar, procesa todas las tramas desde que se paró hasta que vuelve a
ponerse en ejecución. Aun así, el proceso es capaz de recuperarse en unos
minutos realizando el procesamiento streaming a su hora. Por otro lado,
hemos comprobado que el rendimiento, añadiendo las consultas a la base de
datos de OSM, penaliza mucho el rendimiento. Realizando dichas consultas,
el procesamiento no es capaz de procesar las 3000 tramas en los 10 segundos
de tiempo que se usan, llegando a tardar más de 40 segundos para
procesarlas. Aunque el proceso es capaz de procesar las tramas, no es capaz
de acompasarse debido al acceso a datos. Aunque existe la posibilidad de
integrar MongoDB con Spark, descartamos dicha opción ya que carga toda la
colección para hacer uso de ella, por lo que no sería eficiente en memoria
en el caso de la base de datos de OSM. Por otro lado, encontramos que es
realmente sencillo de usar, ya que con un solo script de Python no hemos
tenido que manejar la concurrencia entre los diferentes servidores y,
además, nos ofrece gran cantidad de operaciones que podríamos encontrar en
una base de datos relacional tradicional.

En cuanto a MongoDB, encontramos que es una herramienta muy potente a la
hora de manejar gran cantidad de datos. Asimismo, encontramos una gran
flexibilidad a la hora de introducir datos que, en bases de datos
relacionales, es muy dificil de conseguir ya que, al manejar gran cantidad
de filas sin indexar, se hacen mucho más lentas. Además, es muy fácil
realizar APIs sobre esta ya que el mismo motor devuelve objetos JSON muy
usados en el mundo web. Por tanto, calificamos esta herramienta como válida
para diferentes propósitos, pero muy especialmente en el nuestro, por la
flexibilidad a la hora de tratar diferentes tipos de datos.

Por último, el Stack de Elastic, nos ha ayudado mucho a ver los datos en
tiempo real. El coste de trabajo de poner estas herramientas en producción
es muy bajo con el rendimiento que obtenemos, además de la facilidad de
usar Kibana para analizar los datos. También decir que esta herramienta nos
permite sabe el trafico que trasncurre en un punto en concreto, pudiendo
hacer uso de la consulta que encontramos en el anexo \ref{apendJ}.

Dicho esto, concluimos que se han validado los requisitos propuestos por la
empresa. Por su parte, Cristóbal, CTO de \mdata{}, nos ha validado el
estudio realizado valorándolo muy positivamente y, de no ser por la compra
de \vzconnect, valoraría la posibilidad de migrar a dicha arquitectura.

%%% Local variables:
%%% TeX-master: "main.tex"
%%% coding: utf-8
%%% ispell-local-dictionary: "spanish"
%%% TeX-parse-self: t
%%% TeX-auto-save: t
%%% fill-column: 75
%%% End:
