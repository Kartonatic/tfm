\chapter{Conclusiones y trabajo futuro}

El objetivo principal de este proyecto ha sido la elección, implementación
y evaluación de una arquitectura Big Data para aplicaciones de gestión de
flotas. Este trabajo se ha definido como una prueba de concepto destinado a
la empresa \mdata{} para dar a conocer a dicha empresa otro tipo de
arquitecturas software y herramientas que ofrece este paradigma. Aunque,
durante el transcurso del proyecto, \mdata{} fuera adquirida por la
multinacional Verizon, este proyecto ha tenido un grato impacto en la
empresa. La arquitectura Lambda ha abierto un nuevo horizonte, ya que
podemos añadir o cambiar herramientas de una forma sencilla. Además,
gracias al conocimiento de nuevas herramientas se podrán proponer nuevos
desarrollos que mejoren la calidad de los servicios ofrecidos.

En cuanto al trabajo realizado, se ha explorado la potencia de una
arquitectura Lambda, siendo capaces de apreciar la flexibilidad de la misma
y la gran capacidad de cómputo que tiene. De esta misma forma, el hecho de
haber usado las herramientas expuestas ha ayudado a entender por qué están
en auge en este momento, entendiendo cómo funcionan y cómo llevarlas a un
plano productivo. Por su parte, que estas herramientas sean escalables
horizontalmente sin tener que modificar el código que encontramos en
producción es lo que hace a estas herramientas tan valiosas. Por un lado,
encontramos Docker que, aunque la curva de aprendizaje es alta al
principio, nos permitirá crear diferentes servicios totalmente portables.
Por otro lado, encontramos Apache Hadoop, que nos proporciona escalar el
almacenamiento, tan solo añadiendo nuevas máquinas de forma transparente al
usuario final. Apache Spark, por su parte, nos permite realizar un
procesamiento en Streaming realmente eficiente pudiendo distribuir el
trabajo en diferentes zonas, incluso haciendo uso del mismo cluster de
Apache Hadoop, todo esto, sin cambiar el código que tengamos en producción.
Apache Kafka, nos proporcionará un sistema de acceso a los datos en
streaming realmente rápido, pudiendo leer los mismos datos con diferentes
procesos según el propósito. Por su parte, MongoDB nos ha proporcionado una
gran cantidad de funciones geoespaciales para realizar consultas y, además
de esto, el hecho de que nos retorne JSON la hace muy apropiada para su uso
en diferentes aplicaciones, principalmente en aplicaciones web. En este
mismo orden, el Stack de Elastic nos proporciona una gran cantidad de
herramientas para controlar los flujos de datos e insertarlos en
Elasticsearch proporcionando, además, una herramienta de visualización
capaz de manejar gran cantidad de datos en tiempo real de una forma muy
cómoda. Para terminar, otra característica importante de estas herramientas
es la cantidad de librerías que tienen para distintos lenguajes, lo que nos
ha facilitado el desarrollo de las mismas para usar el mismo lenguaje de
programación en cada una de ellas.

Concluimos con que este paradigma encaja muy bien en este tipo de
aplicaciones, en concreto, la arquitectura Lambda, que nos proporciona la
capacidad de procesamiento en dos líneas que necesitan las empresas del
sector. Además, las herramientas elegidas son más que aptas para aumentar
la cantidad de datos que se recogen permitiendo, de igual manera, aumentar
la capacidad de cómputo fácilmente.

En cuanto a los inconvenientes encontrados, el principal fue el tener que
desarrollar dicha arquitectura sobre un hardware reducido. El hecho de usar
dicho hardware ha implicado no tener un tiempo de respuesta óptimo ya que,
debido a que los diferentes servicios tienen que estar esperando su slot de
CPU y reservar su espacio de memoria, la simulación del cluster se ha visto
afectada. Por otro lado, aunque menor, el hecho de que Apache Spark no
tenga tipos y funciones geográficas para procesar los datos ha supuesto la
implementación de las mismas, dejando de lado la orientación de los
vehículos a la hora de procesarlos. Otro de los inconvenientes encontrados
ha sido el hecho de tener que realizar peticiones a una base de datos
externa en tiempo real, en nuestro caso a MongoDB. Tener que realizar gran
cantidad de consultas geográficas ralentiza mucho el proceso, implicando
que tengamos que barajar la posibilidad de modificar el proceso y pasarlo a
la Batch Layer.

Una de las mayores dificultades presentadas en este proyecto fue la curva
de aprendizaje de Docker. Dicha herramienta es costosa de aprender ya que
tiene una gran cantidad de opciones a la hora de montar diferentes
imágenes. La fase más costosa del proyecto fue la de montar Hadoop dado
que, al estar aprendiendo Docker, montar un cluster que se comunicara
realizando la redirección de direcciones de red, no fue tarea fácil. Por
otra parte, supuso una dificultad que las contenedores no tuvieran
persistencia por defecto, ya que se debían crear volúmenes o mapear los
directorios con los del sistema operativo. Dicho esto, una vez aprendido,
me ha resultado más ágil de usar que creando máquinas virtuales.

Además de esto, encontramos que, gracias al Máster Inter-Universitario en
Tecnologías de Análisis de Datos Masivos, he sido capaz de realizar, de
manera ágil y efectiva, el código necesario para llevar a cabo la pequeña
aplicación de prueba. Gracias a los conceptos aprendidos en este máster, la
búsqueda de herramientas y poder entender rápidamente los conceptos que se
presentan en la documentación de las herramientas, el trabajo se ha podido
llevar a cabo de una forma satisfactoria.

Por último, comentar algunas futuras mejoras que se realizarán sobre esta
prueba de concepto. Como primera mejora sería migrar nuestra plataforma a
Kubernetes comprobando cómo se comporta, de forma que introducir nuevas
máquinas, sea más fácil. Por otro lado, también se pretende introducir la
arquitectura en un cluster, valorando entre los diferentes precios que se
nos proponen en el mercado, para ver cómo se comporta la plataforma en un
cluster real. En cuanto a la funcionalidad, se pretende añadir la dirección
a la que va el vehículo para mejorar la detección de llegada a un punto.
Además de esto, se pretende introducir un sistema de alarmas capaz de
comunicar, en tiempo real, incidencias vía SMS, correo electrónico o con
una notificación de una aplicación. En cuanto a la parte de detección de
exceso de velocidad, se pretende realizar pruebas con un cluster
distribuido de MongoDB, y cambiar el código de integración de mapas para
que indexe los puntos geográficos mejorando el tiempo de respuesta. Por
último, se pretende introducir el flujo de datos en las colas de Kafka y
realizar el procesamiento.

%%% Local variables:
%%% TeX-master: "main.tex"
%%% coding: utf-8
%%% ispell-local-dictionary: "spanish"
%%% TeX-parse-self: t
%%% TeX-auto-save: t
%%% fill-column: 75
%%% End:
