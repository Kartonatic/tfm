\chapter{SparkUI\label{spark-ui}}

Durante la ejecución de una aplicación Spark, el \texttt{SparkContext}
correspondiente despliega una aplicación web con el primer puerto
disponible a partir del~\texttt{4040}~\cite{prosparkstreaming}. La web
mostrada nos ofrece información muy útil para comprender el funcionamiento
de la aplicación. Explicaremos sus componentes principales analizando
nuestra aplicación de inferencia de esquemas.

El aspecto más importante, una vez nuestra herramienta esté funcionando, es
comprobar que sea \emph{estable} en la pestaña \emph{Streaming} de la web,
Las figuras~\ref{figure:ui-unstable} y~\ref{figure:ui-stable} describen
escenarios en los que nuestra aplicación de inferencia se comporta de
manera \emph{inestable} y \emph{estable} respectivamente. Spark Streaming
nos ofrece la opción de configuración
\texttt{spark.streaming.backpressure.enabled} mediante la que delegamos en
Spark el control del ratio de recepción~\cite{spark-configuration}, lo que
causará que sea el emisor de los datos quien los almacene hasta su
procesamiento~\cite{mastering-apache-spark}.

\begin{figure}[htb!]
\centering
%\includegraphics[width=\textwidth]{ui-unstable}
\caption{Aplicación funcionando de manera \emph{inestable}. El tiempo de
  procesamiento \emph{supera} intervalo de batch con regularidad, lo que
  causa retrasos en el procesamiento de nuevos
  lotes.\label{figure:ui-unstable}}
\end{figure}

%%% Local variables:
%%% TeX-master: "memoria.tex"
%%% coding: utf-8
%%% ispell-local-dictionary: "spanish"
%%% TeX-parse-self: t
%%% TeX-auto-save: t
%%% fill-column: 75
%%% End:
